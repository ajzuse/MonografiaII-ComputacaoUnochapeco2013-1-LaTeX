\section{Estrutura do Trabalho}
No primeiro capítulo deste trabalho é apresentada uma introdução ao trabalho, descrevendo o tema escolhido e os objetivos a serem alcançados. Também são descritos os procedimentos metodológicos utilizados no desenvolvimento do mesmo.

No segundo capítulo deste trabalho são apresentados os conceitos das tecnologias aplicadas neste trabalho, apresentando os sistemas operacionais para dispositivos móveis e as linguagens de programação e bibliotecas utilizadas no desenvolvimento do trabalho.

No terceiro capítulo deste trabalho são apresentadas as diversas categorias de ferramentas de desenvolvimento para dispositivos móveis, explicando seu funcionamento e processo de geração da aplicação.

No quarto capítulo deste trabalho são apresentos os meios de comunicação aplicados no trabalho, explicando o funcionamento das redes sem fio e da telefonia móvel, explicando as diversas tecnologias de telefonia móvel existentes.

No quinto capítulo deste trabalho é apresentada uma visão geral do sistema acadêmico Minha Uno, com informações sobre os seus perfis de acesso.

No sexto capítulo deste trabalho são apresentados os dados da pesquisa feita com os usuários do sistema acadêmico sobre a utilização de dispositivos móveis e a importância dos módulos do sistema acadêmico.

No sétimo capítulo é explicado o desenvolvimento do servidor utilizado para a extração das informações do sistema acadêmico Minha Uno, e também como as informações são disponiblizadas para a aplicação móvel.

No oitavo capítulo é explicado o desenvolvimento da aplicação para dispositivos móveis, demonstrando a forma de desenvolvimento utilizada e o resultado apresentado na aplicação.

No nono capítulo é feita a conclusão do trabalho, e são sugeridos temas para a continuidade do mesmo posteriormente.