\section{Estrutura do Trabalho}
%No primeiro capítulo é apresentada uma introdução ao trabalho, descrevendo o tema escolhido e os objetivos a serem alcançados. Também são descritos os procedimentos metodológicos utilizados no desenvolvimento do mesmo.

No capítulo 2 são apresentados os conceitos das tecnologias aplicadas neste trabalho, apresentando os sistemas operacionais para dispositivos móveis e as linguagens de programação e bibliotecas utilizadas no desenvolvimento do trabalho.

No capítulo 3 são apresentadas as diversas categorias de ferramentas de desenvolvimento para dispositivos móveis, explicando seu funcionamento e processo de geração da aplicação.

No capítulo 4 são apresentados os meios de comunicação aplicados no trabalho, explicando o funcionamento das redes sem fio e da telefonia móvel, explicando as diversas tecnologias de telefonia móvel existentes.

No capítulo 5 é apresentada uma visão geral do sistema acadêmico Minha Uno, com informações sobre os seus perfis de acesso.

No capítulo 6 são apresentados os dados da pesquisa feita com os usuários do sistema acadêmico sobre a utilização de dispositivos móveis e a importância dos módulos do sistema acadêmico.

No capítulo 7 é apresentado o desenvolvimento do servidor utilizado para a extração das informações do sistema acadêmico Minha Uno, e também como as informações são disponiblizadas para a aplicação móvel.

No capítulo 8 é apresentado o desenvolvimento da aplicação para dispositivos móveis, demonstrando a forma de desenvolvimento utilizada e o resultado apresentado na aplicação.

O capítulo 9 consiste na conclusão do trabalho, e são sugeridos temas para a continuidade do mesmo posteriormente.

No apêndice A é apresentada a implementação do servidor de extração desenvolvido.

No apêndice B é apresentada a implementação da aplicação desenvolvida.

No apêndice C é apresentada a implementação do servidor REST utilizado, com uma alteração para o correto funcionamento.