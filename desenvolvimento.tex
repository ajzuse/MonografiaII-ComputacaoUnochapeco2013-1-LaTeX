\section{Estrutura do Trabalho}
No primeiro capítulo deste trabalho é apresentada uma introdução ao trabalho, descrevendo o tema escolhido e os objetivos a serem alcançados. Também são descritos os procedimentos metodológicos utilizados no desenvolvimento do mesmo.

No segundo capítulo deste trabalho são apresentados os conceitos utilizados no trabalho, explicando as tecnologias, linguagens de programação utilizadas e os tipos de ferramentas de desenvolvimento multiplataforma. Também é apresentada uma pequena introdução as redes sem fio e as redes de telefonia móvel.

No terceiro capítulo deste trabalho é apresentada uma visão geral do sistema acadêmico Minha Uno, com informações sobre os seus perfis de acesso.

No quarto capítulo deste trabalho são apresentados os dados da pesquisa feita com os usuários do sistema acadêmico sobre a utilização de dispositivos móveis e a importância dos módulos do sistema acadêmico.

No quinto capítulo é explicado o desenvolvimento do servidor utilizado para a extração das informações do sistema acadêmico Minha Uno, e também como as informações são disponiblizadas para a aplicação móvel.

No sexto capítulo é explicado o desenvolvimento da aplicação para dispositivos móveis, demonstrando a forma de desenvolvimento utilizada e o resultado apresentado na aplicação.

No sétimo capítulo é feita a conclusão do trabalho, e são sugeridos temas para a continuidade do mesmo posteriormente.