\chapter{Revisão Bibliográfica}

\section{Tecnologias}

\subsection{Sistema Operacional para Dispositivos Móveis}
Sistemas operacionais para dispositivos móveis seguem o mesmo conceito aplicado a outras plataformas, que segundo Tanembaum, é de difícil definição por meio de um único conceito, pois o mesmo é responsável pelo Gerenciamento do Hardware hospedeiro do mesmo, como também fornecer um conjunto de recursos abstratos claros em vez de recursos confusos de hardware aos aplicativos \cite{tanenbaum2010modern}. Em outras palavras, pode-se dizer que o sistema operacional é formado de um conjunto de softwares e bibliotecas que tornam mais fácil a interação entre a máquina e os usuários.

\subsubsection{Symbian}
Desenvolvido pela parceria entre Nokia, Ericsson, Motorola e PSION, foi usado amplamente pela Nokia em praticamente todos os seus dispositivos no passado. O sistema operacional tinha como foco a integridade e segurança de dados, evitar desperdício de tempo do usuário e trabalhar com recursos escassos. Era reconhecido pelo gerenciamento extremamente eficiente de recursos como bateria, processador e memória\cite{SistemasOperacionaisMoveisComputacao}.

Devido a concorrência de outros sistemas operacionais para dispositivos móveis, e por ter uma diminuição significativa nas suas vendas, o Symbian foi descontinuado pela Nokia, a principal investidora e utilizadora destes sistema operacional, anunciando em janeiro de 2013 a não utilização do sistema em seus aparelhos, substituindo o mesmo pelo Windows Phone nos smartphones da companhia \cite{NokiaSymbian}.

\subsubsection{iOS}
Desenvolvido pela Apple, tendo como base o sistema operacional Mac OS X, o iOS inicialmente foi desenvolvido para iPhone, depois expandido para outros produtos como iPad e iPod. É o único sistema operacional analisado desenvolvido para dispositivos específicos, sendo totalmente fechado aos mesmos. Apresenta um ótimo desempenho devido a integração entre Hardware e Software, além da interface gráfica intuitiva, porém apresenta dificuldades para interagir com outros dispositivos externos (como por exemplo outros celulares utilizando a tecnologia Bluetooth) \cite{AvaliacaoComparativaSOMoveis}.

Atualmente o iOS encontra-se na sua sexta versão, e mantém as suas principais qualidades desde a primeira versão, sendo destacadas a interface gráfica intuitiva, recursos, facilidade da utilização e estabilidade do sistema. Além disso, é possível atualizar facilmente um dispositivo iOS quando uma nova versão é lançada, sendo todo processo feito pelo próprio dispositivo caso o usuário assim deseje \cite{Apple}.

Devido ao fato de a Apple produzir tanto o hardware como o software, o sistema operacional é totalmente compatível com o hardware a ser instalado, disponibilizando os recursos conforme o hardware de cada dispositivo. Além disso, a Apple preocupa-se com a segurança e estabilidade do sistema, implementando meios de proteção física contra vírus e malwares, e via aplicação para proteger os dados do usuários \cite{Apple}.

\subsubsection{Google Android}
O Google Android surgiu da necessidades de várias empresas fabricantes de Hardware, Software e Operadoras de Telefonia Móvel de um sistema operacional que apresentasse vantagens para os desenvolvedores de Software e uma experiência de usuário inovadora, com muitos recursos sem abrir mão da beleza e da facilidade de uso. Desenvolvido pela Open Handset Alliance \cite{OHA}, o Google Android é um sistema operacional desenvolvido sobre o kernel Linux com aplicações nativas e a possibilidade desenvolvimento de aplicações por qualquer pessoa que possua as habilidades e vontade para isso \cite{lechetagoogle}.

Atualmente o Google Android encontra-se na sua versão 4 revisão 2, e apresenta como novidades desta versão a possibilidade de os usuários do Sistema Operacional que utilizam o mesmo em Tablets possam criar perfis diferentes, com aplicativos e configurações diferentes. Além disso o Google trouxe ao Android recursos de transmissão de imagens sem fio e melhorias no desempenho do sistmea operacional em geral. Para completar foram implementados novos recursos para a câmera e também um teclado melhorado \cite{GoogleAndroid}.

\subsubsection{Windows Phone}
Lançado pela Microsoft em 2010, como sucessor do Windows Mobile, o Windows Phone foi desenvolvido para atender tanto usuários comuns como corporativos. O Windows Phone foi desenvolvido para rodar em diversos hardwares, podendo ser licenciado pelas fabricantes que desejarem usar o mesmo em seus dispositivos. Possui integração total com o ambiente .NET também da Microsoft, além das tecnologias de desenvolvimento Silverlight e XNA. Apresenta uma desvantagem em relação aos outros sistemas operacionais que é não possuir multithread, permitindo que apenas uma aplicação seja executada por vez no dispositivo \cite{AvaliacaoComparativaSOMoveis}.

\subsection{Web service}
Web service é um sistema de software que permite a comunicação entre diversos softwares por meio de mensagens SOAP, normalmente utilizando o Protocolo de transferência de Hipertexto (HTTP) para transmitir arquivos de dados entre as aplicações \cite{W3C}.
\subsubsection{REST}
A arquitetura de Transferência de Estado Representativo (do inglês \emph{REpresentational State Transfer}, também identificado como REST) é uma técnica da engenharia de software que define uma interface uniforme de conexão. O REST é um estilo híbrido, baseado em muitos protocolos baseados em rede (como Protocolo Cliente-Servidor, Protocolo de Sistemas em Camadas, Protocolo de Sessão Remota, etc.) \cite{RESTFielding}.

A arquitetura REST funciona com o conceito Cliente-Servidor, onde as informações são disponibilizadas em um servidor, que responde de forma padronizada as requisições de informações, retornando com um mesmo layout os dados retornados. Isto permite que várias aplicações venham a consumir informações do mesmo servidor, porém mostrem estas informações de diferentes formas, já que o servidor REST não retorna componentes gráficos, apenas texto em algum padrão definido (exemplo XML, JSON, Texto delimitado por caractere) \cite{RESTFielding}.

Um servidor REST é um servidor sem estado, ou seja, as informações retornadas pelo servidor não são armazenadas no mesmo. Desta forma a visibilidade, confiabilidade e escalabilidade do servidor são melhoradas. A melhoria da visibilidade ocorre pois o servidor não precisa olhar para outros estados em busca de dados, olhando apenas a requisição feita para o mesmo e retornando as informações referentes a ela. Já a confiabilidade melhora pois por ser sem estado o servidor consegue recuperar-se de falhas parciais de forma automática, onde ao se repetir a requisição o erro ocorrido anteriormente não influencia uma nova consulta. Já a escalabilidade é melhorada pois os recursos alocados no servidor para cada consulta são desalocados rápidamente após o termino da mesma, não mantendo nenhuma informação daquela consulta em memória temporária ou permanente. A desvantagem de ser uma arquitetura sem estado é que requisições repetitivas de informações sempre serão buscadas na fonte dos dados, podendo gerar sobrecarga na rede ou em servidores para tratamento destas requisições.\cite{RESTFielding}.

Além disso um servidor REST tem como uma das suas principais características a padronização na disponibilização das informações, onde aplicando-se os conceitos da engenharia de software de generalização é possível obter de diferentes servidores REST que possuem como base o mesmo tipo de retorno informações diferentes sem alterar-se a forma de obtenção das informações. Desta forma uma vez desenvolvido o servidor, o consumo das informações ocorre de forma rápida e simples pela aplicação, por seguir um padrão já conhecido de retorno. Este padrão de retorno, conforme comentado anteriormente, não ocorre de forma estática, e é desenvolvido pela implementação do servidor \cite{RESTFielding}.

\section{JavaScript}
Apesar do nome possuir a palavra \emph{Java}, o JavaScript não possui ligação nenhuma com esta linguagem de desenvolvimento, sendo o nome escolhido desta forma apenas para causar a confusão e a curiosidade dos desenvolvedores. O JavaScript é uma linguagem de programação dinâmica e orientada a objetos para propósitos gerais, sendo seu principal uso no desenvovimento de scipts para a internet. Atualmente qualquer computador possui um interpretador JavaScript instalado, seja ele integrado ao navegador web ou uma aplicação externa \cite{JavaScriptCrockford}.

O JavaScript foi desenvolvido inicialmente pela Netscape com o nome de LiveScript, porém o nome nõa era confuso o suficiente, sendo alterado para JavaScript posteriormente. A sintaxe das aplicações desenvolvidas em JavaScript lembram de uma aplicação desenvolvida em Java, e consequentemente de uma aplicação desenvolvida em C, porém o JavaScript não possui estruturas de tipos, e apesar de suportar orientação a objetos também não possui classes, o que também mostra que não possuí herança por classes, porém possui herança orientada a protótipos \cite{JavaScriptCrockford}.

O fato de o JavaScript ser uma linguagem com uma curva de aprendizado suave faz com que ele seja utilizado em meio acadêmico para aprendizagem, o que faz com que a linguagem seja conhecida pela sua simplicidade, não pelo seu potencial. Além disso a especificação da linguagem feita pela \emph{ECMA} (Europea Computer Manufactures Association) de uma qualidade muito baixa e de dificil leitura auxiliou no mal entendimento do funcionamento e das especificações da linguagem. Unindo este fato a livros escritos por pessoas que não conhecem a fundo a linguagem e assim demonstram exemplos que não seguem as melhores práticas de desenvolvimento, gera-se um mito de que o JavaScript é uma linguagem desorganizada e que não existe um padrão.\cite{JavaScriptCrockford}.

\subsection{JSON}
Baseando-se em um subconjunto da linguagem JavaScript, o JSON (Notação de Objetos JavaScript, do inglês JavaScript Object Notation) é um formato leve para troca de dados. Sendo completamente independente da linguagem JavaScript, apenas apresentando similaridades com a mesma, é um formato familiar a maioria das linguagens de programação modernas, sendo assim considerado o formato ideal para troca de dados. O JSON representa basicamente dois tipos de estruturas de dados, objetos e vetores.\cite{JSON}.

Os objetos são representados entre chaves (iniciando em \{ e terminando em \}) e representando os atributos no formato chave-valor, onde a chave obrigatóriamente apresenta-se entre aspas (") e é separada do valor por dois-pontos (:). Os valores podem ser do tipo vetor, objeto, booleano, nulo, texto e número. Todos os atributos de um objeto são separados por vírgula (,), sendo que o último atributo da classe não leva vírgula antes do fechamento das chaves. Um exemplo de objeto JSON é pode ser visto na Figura 1 \cite{JSON}.

\begin{figure}[!htb]
     \centering
     \caption{Exemplo de Objeto JSON}
     \includegraphics[scale=1]{imagens/exemplojson.png}
     \\ Fonte: \cite{JSONLint}.
\end{figure}

Na Figura 1 também é exibido um vetor, representado pelo atributo \emph{sub1} do objeto \emph{sub}. Como pode-se ver, um vetor tem como característica estar entre colchetes (iniciando em [ e terminando em ]), e seus objetos ou valores separados por vírgula (,), exceto o último valor antes dos colchetes serem fechados \cite{JSON}.

\section{Java}

Criado por James Gosling a partir de um projeto de pesquisa da Sun Microsystem em 1991, chamado inicialmente de Oak e posteriormente renomeado para Java pois o nome Oak já pertencia a uma linguagem de programação.\cite{java_deitel}.

Baseada na linguagem C++ e orientada a objetos, foi criada para ser utilizada em dispositivos eletrônicos, porém com a expansão da \emph{Web} a Sun viu o potencial de utilizar o Java para adicionar conteúdo dinâmico as páginas \emph{Web}. A linguagem foi anunciada oficialmente em 1995 em uma feira do setor \cite{java_deitel}.

\section{SQLite}
O SQLite é uma biblioteca de código fonte livre compilada juntamente com a aplicação que implementa um controlador de base de dados SQL para uso geral. Por funcionar embutida na aplicação, não necessita de um servidor ou configurações para funcionar, bastando definir o nome do arquivo da base e a biblioteca se encarrega dos controles necessários \cite{SQLite}.

Apesar de todas as funcionalidades fornecidas pelo SQLite, como a possibilidade de execução de consutas SQL, gerenciamento de transações e múltiplas tabelas, indices e gatilhos, o SQLite é econômico em termos de memória utilizada, sendo sua utilização recomendada em dispositivos com pouco espaço de armazenamento ou memória ram, como celulares, players mp3 entre outros. Mesmo neste tipo de ambientes, o SQLite apresenta um bom desempenho \cite{SQLite}.

\section{jsoup}
O jsoup é uma biblioteca para a linguagem java que fornece uma API para extração e manipulação de dados a partir de código HTML utilizando o melhor do DOM, CSS e métodos jquery para extrair as informações \cite{JSOUP}.

O jsoup permite a extração das informações acessando o código HTML a partir de uma URL, um arquivo ou de um texto, sendo que para a busca e extração das informações podem ser utilizados DOM ou seletores CSS. Além disso o jsoup permite manipular os elementos, atributos e textos dos códitos HTML \cite{JSOUP}.

O jsoup foi desenvolvido para permitir a extração dos mais diferentes formatos de código HTML disponíveis, validando o HTML de entrada e criando uma árvore informações que posteriormente são extraídas conforme a expressão indicada na extração\cite{JSOUP}.

\section{Ferramentas de Desenvolvimento Multiplataforma para Dispositivos Móveis}
Um dos grandes desafios para os desenvolvedores de software é encontrar qual ferramenta atende as necessidades da aplicação a ser desenvolvida. Isso fica ainda mais perceptível quando a aplicação deve ser executada em dispositivos móveis, devido a diversidade de sistemas operacionais a serem atendidos. Hoje existem 4 sistemas operacionais para dispositivos móveis que detém grande parte do mercado, sendo eles Android, iOS, Windows Phone e Symbian, sendo o Android e o iOS detentores da maior fatia do mercado. Ao se deparar com este cenário, desenvolvedores se perguntam qual tipo de ferramenta permite que eles desenvolvam apenas uma vez a aplicação, e que a mesma rode de forma satisfatória em todos os dispositivos. Abaixo será visto as 3 principais técnicas de desenvolvimento multiplataforma para dispositivos móveis\cite{CrossPlatformMobileDevelopment2011}.

\subsection{Compilação Cruzada}
Ferramentas do tipo Compilação Cruzada separam os ambientes de compilação conforme a plataforma de destino, gerando código fonte nativo para a plataforma de destino neste momento do processo de geração de um aplicativo. Com isso são geradas aplicações que são nativas para as plataformas alvo. Este tipo de ferramenta possui a desvantagem de consumir mais tempo durante a compilação, devido a conversão de código, e por apresentar certa demora para aderir a novas plataformas
\cite{CrossPlatformMobileDevelopment2011}.

\subsubsection{Titanium}
Desenvolvido pela Appcelerator, o Titanium foi lançado em dezembro de 2008. Baseado na técnica de compilação cruzada, o desenvolvedor utiliza-se de uma API JavaScript para o desenvolvimento da aplicação, e no momento da compilação são gerados códigos nativos para a diferentes plataformas. A compilação é composta por 3 etapas, sendo elas: Pré compilação (Otimização do código JavaScript), Compilação de Front-End (Geração de código nativo da plataforma alvo) e Compilação e Empacotamento para a plataforma (Geração do aplicativo). A diferença entre o Titanium e os outros frameworks descritos abaixo é a geração de interfaces que não utilizam de um motor de navegação, mas que são nativas, geradas por código nativo resultante da compilação do código JavaScript
\cite{CrossPlatformMobileDevelopment2011}.

O Appcelerator Titanium suporta, a partir do mesmo código fonte, a compilação de aplicativos nativos para iOS, Android, Blackberry, Windows e além disso gera páginas web para outros dispositivos móveis. Além do fato de todas estas plataformas serem contempladas pelo titanium, o fato de ser utilizado o JavaScript como linguagem de desenvolvimento faz com que muitos desenvolvedores adotem o Titanium como SDK de desenvolvimento. Além disso, a utilização de uma ferramenta única permite que seja mais fácil manter os projetos de aplicativos móveis multiplataforma, onde a alteração é única para todas as plataformas \cite{appceleratorTitanium}.

Atualmente o Titanium possui mais de 468,667 desenvolvedores contribuindo para o projeto, melhorando os recursos existentes e desenvolvendo novos recursos. Este número é formado por desenvolvedores independentes, parceiros da appcelerator ou funcionários, que desenvolvem novos módulos e extensões para melhorar e facilitar o desenvolvimento das aplicações pelos utilizadores do SDK \cite{appceleratorTitanium}.

\subsection{Máquina Virtual}
Ferramentas do tipo Máquina Virtual ficam entre as ferramentas baseadas em Web e as ferramentas de Compilação Cruzada. Nelas o código-fonte é empacotado juntamente com bibliotecas e uma máquina virtual, que em tempo de execução converte os comandos do código fonte em comandos nativos da plataforma que está executando a aplicação. Sua desvantagem é que normalmente aplicações maiores apresentam um certo delay na sua execução, devido a conversão do código
\cite{CrossPlatformMobileDevelopment2011}.

\subsubsection{Rhodes}
Lançado em 2008, o Rhodes é um framework para desenvolvimento que faz parte do ecossistema de ferramentas voltadas para o desenvolvimento multiplataforma. Por se utilizar de uma Máquina virtual para a execução dos programas desenvolvidos, o mesmo converte as aplicações desenvolvidas utilizando Ruby na sua parte lógica, e \emph{HyperText Markup Language} (HTML), JavaScript e \emph{Cascading Style Sheets} (CSS) para a interface gráfica para código nativo da plataforma em que a aplicação está sendo executada, apresentando uma experiência de usuário similar a de aplicações desenvolvidas em código nativo para a plataforma em que a aplicação está sendo executada
\cite{CrossPlatformMobileDevelopment2011}.

\subsection{Web}
Ferramentas do tipo Web empacotam código fonte voltado para a internet (normalmente desenvolvidos em HTML e CSS) juntamente com um Webkit, dando a impressão para o usuário que se trata de uma aplicação desenvolvida para aquela plataforma. Possui como principal vantagem o desenvolvimento simplificado porém , por não utilizar componentes gráficos do sistema operacional hospedeiro, apenas simula o comportamento de uma aplicação nativa. Além disso, depende dos recursos fornecidos pelo webkit do sistema operacional para determinar o nível de acesso a funcionalidades do mesmo 
\cite{CrossPlatformMobileDevelopment2011}.

\subsubsection{PhoneGap}
Criado no inicio de 2008, o PhoneGap fornece um conjunto de ferramentas para desenvolvimento de aplicações multiplataforma para dispositivos móveis utilizando apenas código HTML, JavaScript e CSS. Muito popular pela sua flexibilidade, arquitetura simples e de fácil utilização. Utilizando um modelo híbrido de execução, um único código-fonte escrito em HTML, JavaScript e CSS são executados em um browser empacotado em forma de aplicação nativa, suportado em cada plataforma de destino. O Acesso as funcionalidades do Sistema Operacional é feito por meio da chamada de métodos JavaScript que fazem as requisições na API proprietária do Sistema Operacional. É uma alternativa para a portabilidade de aplicações Web para dispositivos móveis, mas deve-se lembrar que não se possui acesso aos componentes gráficos nativos do sistema operacional
\cite{CrossPlatformMobileDevelopment2011}.

\section{Comunicação}

\subsection{Redes Sem Fio}
Criadas para substituirem as redes cabeadas em locais onde o cabeamento não é possível, as redes sem fio (do inglês \emph{Wireless}) utilizam frequências de rádio para a transmissão das informações \cite{RedesSemFioUFLA}.

As redes sem fio tem como principal característica não haver necessidade de cabos para conectar vários dispositivos. Este tipo de rede apresenta uma complexidade maior comparada com as redes cabeadas, devido a mobilidade dos dispositivos, o que não ocorre com os dispositivos utilizados nas redes com fio, onde a mobilidade máxima é a permitida pelo cabo ao qual o dispositivo está conectado. Além disso, neste tipo de rede podem ocorrer problemas de interferência entre redes que compartilham o mesmo espaço físico \cite{IEEE80211}.

Apesar das discussões sobre a confiabilidade e a segurança das redes sem fio, existe um consenso de que a configuração fácil, flexibilidade e gerenciamento das redes sem fio fazem com que a mesma sejam uma opção viável quando existe a mobilidade dos dispositivos ou que diferentes dispositivos conectem-se temporariamente a rede. Além disso, graças aos padrões e protocolos é possível que as redes sem fio comuniquem com as redes cabeadas, permitindo a troca de informações entre as mesmas \cite{RedesSemFioUFLA}.

\subsection{Telefonia e Internet Móvel}
Desde a primeira geração de telefones móveis introduzidas no mundo durante os anos 80 e 90  até a atual tecnologia 4G (LTE) muitas evoluções ocorreram na forma como os celulares se comunicam. A evolução causada por estas tecnologias mudou a forma em que o mundo se comunica, trazendo a possibilidade de conectar-se a internet utilizando um dispositivo móvel onde quer que você esteja, bastando haver sinal da operadora de telefonia.

\subsubsection{1G}
As redes de 1ª Geração (1G) causaram grande impacto na sociedade pelo seu nível de inovação. Trabalhando de forma analógica utilizando-se de modulação de frequência (do inglês Frequency Modulation - FM), onde se transmitia a voz do usuário em faixas de Frequência Muito Alta (do inglês Ultra High Frequency - UHF) \cite{AEvolucaoTelefoniaCelular}. Este tipo de rede permite apenas o tráfego de voz, onde a qualidade das ligações varia conforme o nível de interferência. Além disso um dos problemas desta tecnologia é a baixa segurança, onde é possível fazer escuta de ligações utilizando-se de sintonizador de rádio e até mesmo utilizar frequências alheias para efetuar ligações  \cite{GeracoesTelefoniaMovel}.

\subsubsection{2G}
Inserida no mercado no início da década de 90, é marcada pela mudança da tecnologia analógica para a tecnologia digital, permitindo assim além de um maior número de ligações simultâneas na rede, o envio de mensagens de texto (\emph{Short Message Service} (SMS)) e a capacidade de transmitir dados em baixa velocidade entre dispositivos utilizando-se da rede \cite{GeracoesTelefoniaMovel}.

\subsubsection{3G}
A terceira geração de redes móveis é considerada um avanço nas redes 2G baseadas na família de normas da União Internacional de Telecomunicações. Trouxe as redes maior capacidade e serviços para os usuários, como conexões de dados a longas distancias com velocidades variando entre 5 e 10Mbps \cite{GeracoesTelefoniaMovel}.

\subsubsection{4G}
Absorvendo todas as características das redes 3G, as redes 4G propõe mais velocidade e uma nova visão de mercado. Este padrão totalmente baseado em endereçamento IP garante velocidades de acesso entre 100Mbps com dispositivos em movimento e 5Gbps para dispositivos em repouso. Como o 4G é totalmente baseado em IP, também torna possível a integração de qualquer dispositivo que utilize desta tecnologia, como web TV's, gadgets. A tecnologia prevê a serviços como envio de Mensagens Multimídia (do inglês \emph{Multimedia Messaging Service} - MMS), Video Chat, TV Móvel, Broadcast de Video Digital além dos serviços básicos como voz e dados. Além disso prevê a a interoperabilidade entre os diversos padrões de rede sem fio \cite{GeracoesTelefoniaMovel}.
