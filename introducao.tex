\chapter{Introdução}

A adoção de dispositivos móveis com sistema operacional Android e iOS chegam a níveis nunca antes vistos na história da tecnologia. Comparada com outras tecnologias recentes, a adoção de dispositivos do tipo smart foi dez vezes mais rápida que a revolução dos Computadores Pessoais nos anos 80, duas vezes mais rápida que a explosão da Internet nos anos 90 e três vezes mais rápida que a adoção de Redes Sociais. Estima-se que, no mundo, existam 640 milhões de dispositivos com iOS e Android em uso durante o mês de Julho de 2012
\cite{flurry2012}.

No período de Julho de 2011 a Julho de 2012, o Brasil apresentou o 3º maior crescimento no número de dispositivos móveis, com um aumento de 220\%. Estima-se que o Brasil possua 13 milhões de dispositivos com iOS e Android ativos, ocupando o décimo lugar no ranking mundial de dispositivos móveis ativos
\cite{flurry2012}.

Em dados da Huawei (fabricante dos equipamentos utilizados pelas operadoras de telefonia móvel) em seu balanço anual, no Brasil houve um aumento de 99\% no número de usuários da tecnologia 3G. Até o mês de Abril do ano de 2012, em relação ao final de 2011, ocorreu um aumento de 31\% no número de usuários e levando-se em consideração o intervalo do primeiro trimestre de 2011 até o primeiro trimestre de 2012, houve um aumento de 112,6\% no número de usuários
\cite{Huawei1T2012}.

Considerando válida a suposição de que a expansão e uso de dispositivos móveis e tecnologia 3G é similar ao do Brasil na região de Chapecó, pode-se verificar pelos números do balanço social 2011 da Fundação Universitária do Desenvolvimento do Oeste (FUNDESTE), instituição mantenedora da Universidade Comunitária da Região de Chapecó (Unochapecó), que o número de pessoas vinculadas a universidade com acesso a estas tecnologias é significativo. Segundo o balanço social 2011, a Unochapecó conta com 947 funcionários sendo destes 540 docentes e 407 técnico-administrativos. Segundo este mesmo documento, a instituição conta com 8031 acadêmicos de graduação e 910 acadêmicos de pós-graduação, totalizando 8941 acadêmicos ao final do ano de 2011 \cite{Fundeste2011}. Atualmente todas estas pessoas, acadêmicos ou funcionários da Universidade utilizam uma página web para acessar as informações do(s) seu(s) perfil(s), chamada “Minha Uno''. Deste universo de quase 9 mil pessoas, pode-se estimar com base nos dados anteriormente vistos que mais de 600 pessoas envolvidas com a universidade possuam smartphones ou tablets.

\section{Tema}
Devido a crescente utilização de dispositivos móveis (tablets e smartphones) no meio acadêmico, e a facilidade de acesso a internet sem fio e móvel, pretende-se com este trabalho melhorar o acesso ao sistema acadêmico da Universidade Comunitária da Região de Chapecó (Unochapecó) nos mesmos, por meio do desenvolvimento de um aplicativo que notifique o usuário sobre novas informações no sistema e também permita a consulta aos dados do sistema conforme suas permissões, além de poder efetuar cadastros.

\section{Delimitação do Problema}
Tendo como ponto de partida o funcionamento atual do Sistema Acadêmico (Minha Uno), existe uma forma de melhorar a forma de acesso em dispositivos móveis por meio de um aplicativo, agregando recursos que auxiliem os usuários do mesmo?

\section{Questões de Pesquisa}
Quais funcionalidades do sistema acadêmico os usuários desejam que esteja presente em seus dispositivos móveis? 

Quais os impactos serão causados na implantação deste novo aplicativo na infraestrutura atual da instituição? 

Qual tecnologia de desenvolvimento multiplataforma para dispositivos móveis se adapta melhor as necessidades de desenvolvimento da aplicação? 

Qual será a forma de comunicação com o sistema acadêmico atual, para a coleta das informações necessárias para o funcionamento nos dispositivos móveis?

Quais as plataformas que serão contempladas com esta aplicação? 

Quais alterações ou adições de informações no sistema acadêmico devem gerar notificações para os usuários?
