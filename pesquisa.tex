\chapter{Questionário}
Com a finalidade de obter informações sobre a utilização de dispositivos móveis no meio acadêmico, e descobrir a importância de cada item presente no sistema acadêmico atual, entre os dias 25 de agosto de 2012 e 26 de setembro do mesmo ano foi realizado questionário virtual com o corpo acadêmico da universidade. 

Ao total 281 pessoas responderam o questionário, sendo obtidas 300 respostas sobre os diferentes perfis do sistema acadêmico. O questionário teve sua estrutura dividida em três partes, sendo a primeira sobre a utilização de smartphones e tablets no meio acadêmico, a segunda parte sobre a relevância de cada item presente no sistema acadêmico atual (Minha Uno) e a terceira parte sobre o interesse de participar dos testes do aplicativo.

Para o questionário, o perfil “Técnico-Administrativo'' foi tratado como “Funcionários'', sendo esta a nomenclatura adotada na avaliação dos dados coletados. 

\section{Utilização de Dispositivos Móveis}
A primeira parte do questionário foi aplicada específicamente para pessoas que possuem dispositivos móveis do tipo smartphone ou tablet (ou ambos), totalizando 126 pessoas (44,84\% dos entrevistados). 

Com o objetivo de descobrir quais os sistemas operacionais dos dispostivos móveis dos participantes do questionário, foi feita a seguinte pergunta: “Qual o sistema operacional do seu smartphone ou tablet'', sendo as alternativas “Android'', “iOS'', “Windows Phone'', “Symbian'', “Outros'', “Não Sei'' e que foi permitido ao usuário marcar mais de uma alternativa. Os resultados obtidos com esta pergunta mostraram que 56,25\% dos participantes utilizam aparelhos com o sistema operacional Android, seguidos por 17,97\% com iOS, 10,16\% com Symbian, 3,91\% com Windows Phone e 1,56\% com Outros Sistemas Operacionais. Pessoas que não sabiam qual o sistema operacional do seu smartphone ou tablet totalizaram 10,16\%.

Com o objetivo de descobrir qual a forma de acesso a internet móvel mais comum entre os participantes do questionário, foi feita a eles a pergunta “Você utiliza o dispositivo móvel para acessar a internet?'' com as opções “Sim, 3G e Wi-fi'', “Sim, Apenas 3G'', “Sim, Apenas Wi-fi'' e “Não'', sendo que o participante poderia selecionar apenas uma das opções. Obteve-se que 46,83\% dos usuários acessam apenas internet Wi-fi dos seus dispositivos. Em segundo lugar, temos os usuários que acessam internet 3G e Wi-fi, com 45,24\% das respostas. Em terceiro lugar aparecem os usuários que não acessam a internet pelos seus dispositivos móveis, com 4,76\% e em quarto lugar os usuários que acessam a internet apenas via 3G, com 3,17\%.

A última questão feita nesta etapa do questionário se referia ao fato de acessar o sistema acadêmico dos seus dispostivos móveis, sendo que 50,79\% dos participantes informaram que não constuma acessar o sistema acadêmico dos seus dispositivos móveis enquanto o restante (49,21\%) efetuam este tipo de acesso. 

A partir dos dados analisados acima, é possível verificar que, a partir do percentual de participantes que possuem dispositivos móveis, e com os dados do balanço social de 2011 da Fundeste (instituição mantentenedora da Unochapecó), considerando apenas o percentual de pessoas que acessam o sistema acadêmico pelos dispositivos móveis atualmente, aproximadamente 2181 pessoas seriam beneficiadas diretamente com o desenvolvimento de uma aplicação para dispositivos móveis, sendo que se considerar o percentual total de usuários de smartphones ou tablets entrevistados, este número sobe para 4434 pessoas beneficiadas.

\section{Relevância de cada item presente no sistema acadêmico atual}
Com o objetivo de descobrir quais opções disponíveis atualmente no sistema acadêmico são relevantes para os usuários, esta parte do questionário foi dividida em quatro subpartes, onde perguntas específicas sobre cada perfil de acesso do sistema acadêmico foram criadas. 

Ao entrar nesta sessão de perguntas, o usuário inicialmente selecionou qual perfil do sistema acadêmico ele faz parte, e posteriormente foi redirecionado as perguntas específicas de cada perfil. Ao fim das perguntas, o mesmo foi direcionado novamente para o formulário de seleção de perfis, para caso possua acesso a mais de um perfil de usuário, poder responder as perguntas referentes aos outros perfis.

Os perfis de interesse do questionário são os perfis referentes aos alunos de Graduação, alunos de Pós-Graduação, Professores e Funcionários, não constando no questionário o perfil Fornecedor ou outros perfis do sistema acadêmico.

Participaram desta etapa do questionário 281 usuários, onde o perfil “Graduação'' obteve 280 respostas, o perfil “Pós-Graduação'' obteve uma resposta, o perfil “Professor'' obteve duas respostas e o perfil “Funcionário'' obteve 17 respostas.

\subsection{Graduação}

Esta parte do questionário foi destinada apenas para estudantes de cursos de Graduação, onde o acadêmico deu notas de 1 a 5 para cada item disponível no sistema acadêmico atual, utilizando-se da escala de Likert, onde  a nota 1 demonstra que o item menos importante para o acadêmico, e a nota 5 representa que o item é importantíssimo.

Na tabela 1 serão representadas as perguntas efetuadas e a média final das respostas. Esta parte da pesquisa contou com 280 respostas, 93,33\% do total de respostas da pesquisa.
\begin{table}[!hbt]
\centering
\caption[Média de Respostas dos Alunos de Graduação]{Média de Respostas dos Alunos de Graduação sobre os itens do sistema acadêmico}
\vspace{3mm}
\begin{tabular}{p{9.5cm}|c}\hline
\textbf{Item} & \textbf{Média Final} \\ \hline
Bolsa de Pesquisa & 2,84 \\ \hline
Disponibilidade de Laboratório de Informática & 2,46 \\ \hline
Entrega de Trabalhos & 3,70 \\ \hline
Histórico & 3,79 \\ \hline
Horário de Aula/Ementas/Requisitos & 4,29 \\ \hline
Horários do Semestre & 4,39 \\ \hline
Material de Apoio/Planos de Ensino & 4,80 \\ \hline
Notas da Graduação & 4,53 \\ \hline
Situação Financeira & 4,24 \\ \hline
\textbf{Média} & \textbf{3,8} \\ \hline
\textbf{Desvio-Padrão} & \textbf{0,79} \\ \hline
\end{tabular}
\\ Fonte: elaboração do autor.
\end{table}

Aplicando-se a distribuição t de Student, representada pela fórmula 
\[
t=\frac{{\overline{X}} - \mu_0}
{\frac{S_c}{\sqrt{n}}}
\]
onde:\\ 
- $\overline{X}$ é a média amostral observada; \\
- $\mu_0$ é a média esperada, sendo esta 3 na pesquisa; \\
- $S_c$ o desvio padrão amostral corrigido, e; \\
- $n$ é o tamanho da amostra analisada.

O desvio-padrão amostral corrigido é calculado por meio da fórmula 
\[
   S_c = \sqrt{\frac{1}{n-1} \sum_{i=1}^n(X_i - \overline{X})^2} 
\]
Que apresenta parâmetros semelhantes aos utilizados na fórmula utilizada no cálculo da distri- \\ buição t de Student, com o acréscimo do $X_i$ que representa o valor observado para o indivíduo $i$ da amostra
\cite{DistroStudent}.

Aplicando-se as fórmulas acima a tabela 1, foram obtidos os valores apresentados na tabela 2.

\begin{table}[!hbt]
\centering
\caption[Avaliação dos dados obtidos - Graduação]{Avaliação dos dados obtidos com as respostas dos alunos de graduação}
\vspace{3mm}
\begin{tabular}{p{5cm}|c|c|c|c}\hline
\textbf{Item} & \textbf{Média} & \textbf{$S_c$} & \textbf{$t$} & \textbf{Índice de Confiança (\%)} \\ \hline
Bolsa de Pesquisa & 2,84 & 1,435 & -1,9158 & 94,35 \\ \hline
Disponibilidade de Laboratório de Informática & 2,46 & 1,351 & -6,6773 & 99,99 \\ \hline
Entrega de Trabalhos & 3,70 & 1,350 & 8,7214 & 99,99 \\ \hline
Histórico & 3,79 & 1,272 & 10,339 & 99,99 \\ \hline
Horário de Aula/Ementas/Requisitos & 4,29 & 1,067 & 20,221 & 99,99  \\ \hline
Horários do Semestre & 4,39 & 0,906 & 25,738 & 99,99 \\ \hline
Material de Apoio/Planos de Ensino & 4,80 & 0,553 & 54,334 & 99,99 \\ \hline
Notas da Graduação & 4,53 & 0,552 & 46,475 & 99,99 \\ \hline
Situação Financeira & 4,24 & 0,903 & 22,899 & 99,99 \\ \hline
\end{tabular}
\\ Fonte: elaboração do autor.
\end{table}

O cálculo do índice de confiança pode ser reproduzido utilizando-se da função INVT (ou similar) do editor de planilhas eletrônicas, sendo que os percentuais apresentados no índice de confiança foram arredondados para duas casas decimais para permitir melhor interpretação do mesmos. 
Devido a utilização da  distribuição t de Student na forma bicaudal, nos casos onde o valor $t$ apresentou-se negativo, o índice de confiança representa o percentual de chances de a resposta ficar abaixo da média em pesquisas feitas posteriormente, enquanto em itens onde o valor $t$ apresentou-se positivo, o percentual de confiança demonstra as chances de em pesquisas posteriores os valores ficarem acima da média.

\subsection{Pós-Graduação}
Esta parte do questionário foi destinada apenas para estudantes de cursos de Pós-Graduação, onde o pós-graduando deu notas de 1 a 5 para cada item disponível no sistema acadêmico atual, utilizando-se da escala de Likert, onde  a nota 1 demonstra que o item menos importante para o acadêmico, e a nota 5 representa que o item é importantíssimo.

Na tabela 3 serão representadas as perguntas efetuadas e a média final das respostas. Esta parte da pesquisa contou com 1 resposta, não sendo assim possível extrair informações conclusivas sobre este perfil do sistema acadêmico.

\begin{table}[!hbt]
\centering
\caption[Média de Respostas dos Alunos de Pós-Graduação]{Média de Respostas dos Alunos de Pós-Graduação sobre os itens do sistema acadêmico}
\vspace{3mm}
\begin{tabular}{p{9.5cm}|c}\hline
\textbf{Item} & \textbf{Média Final} \\ \hline
Ementas & 3,00 \\ \hline
Entrega de Trabalhos & 5,00 \\ \hline
Material de Apoio/Planos de Ensino & 5,00 \\ \hline
Notas da Pós-Graduação & 5,00 \\ \hline
Situação Financeira & 5,00 \\ \hline
\textbf{Média} & \textbf{4,60} \\ \hline
\textbf{Desvio-Padrão} & \textbf{0,89} \\ \hline
\end{tabular}
\\ Fonte: elaboração do autor.
\end{table}

Devido a possuir apenas uma resposta, não foi efetuada avaliação de Student sobre este item para medir o índice de confiança da resposta obtida, pois o tamanho da amostra pode ser considerada insignificante para a tomada de decisões.

\subsection{Professor}
Esta parte do questionário foi destinada apenas para o corpo docente da instituição onde o professor deu notas de 1 a 5 para cada item disponível no sistema acadêmico atual, utilizando-se da escala de Likert, onde  a nota 1 demonstra que o item menos importante para o acadêmico, e a nota 5 representa que o item é importantíssimo.

Na tabela 4 serão representadas as perguntas efetuadas e a média final das respostas. Esta parte da pesquisa contou com 2 respostas, não sendo assim possível extrair informações conclusivas sobre este perfil do sistema acadêmico.

\begin{table}[!hbt]
\centering
\caption[Média de Respostas dos Professores]{Média de Respostas dos Professores sobre os itens do sistema acadêmico}
\vspace{3mm}
\begin{tabular}{p{9.5cm}|c}\hline
\textbf{Item} & \textbf{Média Final} \\ \hline
Componentes Curriculares Ministrados & 1,50 \\ \hline
Diário de Classe Online & 5,00 \\ \hline
Documentos Diversos & 1,00 \\ \hline
Entrega de Trabalhos & 4,50 \\ \hline
Folha de Pagamento & 2,00 \\ \hline
Gastos dos Convênios Asser & 1,50 \\ \hline
Horários de Aula/Ementas/Requisitos & 4,50 \\ \hline
Horários do Professor & 4,00 \\ \hline
Ligações Telefônicas & 2,00 \\ \hline
Material de Apoio & 5,00 \\ \hline
Período de Férias & 1,00 \\ \hline
Plano de Ensino & 4,00 \\ \hline
Processo Seletivo & 2,00 \\ \hline
Programa de Aprendizagem & 3,00 \\ \hline
Ramais & 3,00 \\ \hline
Registro de Atividades Mensais & 2,50 \\ \hline
Sistema de Mensagens Integrada & 4,00 \\ \hline
\textbf{Média} & \textbf{2,97} \\ \hline
\textbf{Desvio-Padrão} & \textbf{1,40} \\ \hline
\end{tabular}
\\ Fonte: elaboração do autor.
\end{table}

\newpage

Além das questões referentes aos itens acima, também foi feita a pergunta \emph{Sobre o item “Diário de Classe Online'', seria interessante a possibilidade de fazer chamadas e registrar notas pelo smartphone ou tablet?} sendo que 100\% dos participantes responderam afirmativamente.

Devido a possuir apenas duas respostas, não foi efetuada avaliação de Student sobre este item para medir o índice de confiança das respostas obtidas, pois o tamanho da amostra pode ser considerada insignificante para a tomada de decisões.

%\newpage

\subsection{Funcionário}
Esta parte do questionário foi destinada apenas para os funcionários da instituição onde o funcionário deu notas de 1 a 5 para cada item disponível no sistema acadêmico atual, utilizando-se da escala de Likert, onde  a nota 1 demonstra que o item menos importante para o acadêmico, e a nota 5 representa que o item é importantíssimo.

Na tabela 5 serão representadas as perguntas efetuadas e a média final das respostas. Esta parte da pesquisa contou com 17 respostas.

\begin{table}[!hbt]
\centering
\caption[Média de Respostas dos Funcionários]{Média de Respostas dos Funcionários sobre os itens do sistema acadêmico}
\vspace{3mm}
\begin{tabular}{p{9.5cm}|c}\hline
\textbf{Item} & \textbf{Média Final} \\ \hline
Cartão-Ponto & 4,29 \\ \hline
Folha de Pagamento & 4,59 \\ \hline
Gastos dos Convênios Asser & 2,24 \\ \hline
Ligações Telefônicas & 3,00 \\ \hline
Período de Férias & 3,41 \\ \hline
Processo Seletivo & 3,35 \\ \hline
Ramais & 3,65 \\ \hline
Sistema de Mensagem Integrada & 3,59 \\ \hline
Súmula de Currículo & 3,12 \\ \hline
Títulos a Receber & 3,41 \\ \hline
\textbf{Média} & \textbf{3,46} \\ \hline
\textbf{Desvio-Padrão} & \textbf{0,66} \\ \hline
\end{tabular}
\\ Fonte: elaboração do autor.
\end{table}

Aplicando-se a distribuição t de Student, explicada na seção 7.2.1, foram obtidos os índices de confiança mostrados na tabela 6.

\begin{table}[!hbt]
\centering
\caption[Avaliação dos dados obtidos - Funcionário]{Avaliação dos dados obtidos com as respostas dos funcionários da instituição}
\vspace{3mm}
\begin{tabular}{p{5cm}|c|c|c|c}\hline
\textbf{Item} & \textbf{Média} & \textbf{$S_c$} & \textbf{$t$} & \textbf{Índice de Confiança (\%)} \\ \hline
Cartão-Ponto & 4,29 & 0,314 & 16,986 & 99,99 \\ \hline
Folha de Pagamento & 4,59 & 0,190 & 34,388 & 99,99 \\ \hline
Gastos dos Convênios Asser & 2,24 & 0,344 & -9,4499 & 99,99 \\ \hline
Ligações Telefônicas & 3,00 & 0,339 & 0 & 0 \\ \hline
Período de Férias & 3,41 & 0,306 & 5,5489 & 99,99 \\ \hline
Processo Seletivo & 3,35 & 0,293 & 4,9738 & 99,98 \\ \hline
Ramais & 3,65 & 0,359 & 7,4393 & 99,99 \\ \hline
Sistema de Mensagem Integrada & 3,59 & 0,306 & 7,927 & 99,99\\ \hline
Súmula de Currículo & 3,12 & 0,292 & 1,662 & 88,5 \\ \hline
Títulos a Receber & 3,41 & 0,282 & 6,0298 & 99,99 \\ \hline
\end{tabular}
\\ Fonte: elaboração do autor.
\end{table}

O item “Ligações Telefônicas'' apresentou indice de confiança igual a 0 pois a média obtida no mesmo é igual a média esperada ($\mu_0$) para o questionário, tornando a parte superior da fórmula apresentada na seção 7.2.1 zero. Sendo assim a média a ser obtida em questionários posteriores pode ser superior ou inferior a obtida neste questionário.

Para os outros itens aplicam-se as regras presentes na seção 7.2.1, o cálculo do índice de confiança pode ser reproduzido utilizando-se de um editor de planilhas eletrônicas, sendo que neste trabalho os percentuais apresentados no índice de confiança foram arredondados para duas casas decimais para permitir melhor interpretação do mesmos. 

Devido a utilização da  distribuição t de Student na forma bicaudal, nos casos onde o valor $t$ apresentou-se negativo, o indice de confiança representa o percentual de chances de a resposta ficar abaixo da média em pesquisas feitas posteriormente, enquanto em itens onde o valor $t$ apresentou-se positivo, o percentual de confiança demonstra as chances de em pesquisas posteriores os valores ficarem acima da média.

\section{Avaliação dos dados coletados}
Avaliando-se os resultados obtidos no questionário, dados estes apresentados anteriormente, observa-se que os perfis utilizados pelos professores e alunos de pós-graduação não possuem informações suficientes para se chegar a conclusões sobre os itens importantes ou não do sistema.
Por outro lado, os perfis utilizados pelos funcionários e estudantes de graduação obtiveram uma quantidade maior de respostas, fornecendo assim dados mais conclusivos sobre a relevancia dos itens presentes nestes perfis.

Conclui-se que para a implementação do aplicativo para dispositivos móveis, deve-se observar os itens que obtiveram maior relevância nestes perfis em que foi possível efetuar uma análise mais detalhada, onde os itens que obtiveram maiores médias terão maior prioridade sobre os itens que obtiveram médias menores. Além disso, os itens que ficaram com suas médias abaixo de 3,00 não serão considerados importantes na etapa inicial de desenvolvimento, sendo implementados apenas caso haja tempo suficiente após os itens com maiores médias serem implementados no aplicativo.

\section{Interesse em participar dos testes da nova aplicação}
Contando apenas com duas perguntas, a etapa final do questionário teve como objetivos levantar interessados em auxiliar nos testes da aplicação, e obter o contato das pessoas interessadas. Tendo como base os pesquisados que possuem smartphone ou tablet, 99,21\% demonstraram interesse em auxiliar nos testes da nova aplicação.

\section{Implementação}
Com base nos dados apresentados neste capítulo, optou-se em serem implementado apenas o perfil dos alunos de graduação, devido ao número de respostas trazer informações mais confiáveis sobre as opções mais utilizadas pelos acadêmicos. Além disso, devido a forma de extração das informações ocorrer de forma manual e depender do layout da página do sistema acadêmico atual, foi optado por implementar os três módulos com melhores médias (Material de Apoio/Planos de Ension, Notas da Graduação e Horários do Semestre), para demonstrar a possibilidade da extração e também possuir informações reais para a aplicação móvel, extraidas diretamente do Sistema Acadêmico Minha Uno.
