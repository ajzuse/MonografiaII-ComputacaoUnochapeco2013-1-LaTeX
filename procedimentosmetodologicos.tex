\section{Procedimentos Metodológicos}

Esta pesquisa ocorre de forma quantitativa, aplicando um questionário defindo para obter as informações relevantes ao desenvolvimento do projeto e implementação. Os métodos utilizados para a elaboração deste projeto serão feitos através de pesquisa bibliográfica, artigos retirados da internet, coleta de dados feita através de questionário e desenvolvimento de um protótipo de acesso ao sistema para dispositivos móveis.

Quanto à coleta de dados, será feito um questionário com alunos, professores e funcionários da Unochapecó, para levantar informações sobre os usuários do Sistema Acadêmico Minha Uno.  Após o término do questionário e interpretação dos dados coletados, serão determinadas quais as funcionalidades do sistema acadêmico são mais relevantes para seus usuários, percentual de usuários que possuem smartphones, tablets ou ambos, os sistemas operacionais móveis mais utilizados pelos mesmos e a principal forma de acesso a internet utilizada em seus dispositivos móveis. A partir destes dados será possível determinar quais as prioridades no desenvolvimento da aplicação e a divisão das etapas de desenvolvimento da mesma, tendo em vista nas etapas iniciais as funcionalidades mais importantes para cada perfil de usuário.

A pesquisa bibliográfica no aspecto de desenvolvimento para dispositivos móveis e integração com webservice, conta com um número considerável de obras e autores. Portanto, este trabalho acadêmico, neste ponto, terá suas necessidades supridas.


