\section{Justificativa}

A utilização de dispositivos móveis vem batendo records de crescimento a cada ano, superando outras inovações tecnológicas consideradas fundamentais nos dias de hoje, como os Computadores Pessoais, a Internet e as Redes Sociais, estimando-se que existam 640 milhões de dispositivos móveis com sistema operacional Android ou iOS sendo utilizados no mundo durante o mês de Julho de 2012. O Brasil apresentou o terceiro maior crescimento na adoção deste tipo de dispositivos entre 2011 e 2012, tendo um crescimento de 220\% no número de usuários, ficando atrás da China com aumento de 401\% e do Chile com aumento de 279\% neste mesmo período. Além disto, o Brasil encontra-se em décimo lugar no ranking de dispositivos móveis com Sistema Operacional Android ou iOS, com 13 milhões de dispositivos ativos, atrás dos Estados Unidos com 165 milhões de dispositivos, China com 128 milhões, Reino Unido com 31 milhões, Coréia do Sul com 28 milhões de dispositivos, Japão com 22 milhões de dispositivos, Alemanha com 19 milhões de dispositivos, França com 17 milhões de dispositivos, Canadá com 16 milhões de dispositivos e Espanha com 13 milhões de dispositivos
\cite{flurry2012}.

Também percebe-se aumento de 26,2\% na utilização de internet banda larga 3G no mundo no período de 2010 até 2011. No Brasil houve um aumento mais significativo, chegando a 99\% no mesmo período, passando de 20,6 milhões de usuários em 2010 para 41,1 milhões de usuários em 2011. Tendo como período para avaliação desde o primeiro trimestre de 2011 até o primeiro trimestre de 2012, percebe-se um aumento de 112,6\% no número de usuários de conexão banda larga 3G no país \cite{Huawei1T2012}.

Apesar do grande aumento no número de usuários destas tecnologias, o país ainda não se encontra saturado de dispositivos móveis, como ocorre, por exemplo, em Singapura, onde 92\% das pessoas entre 15 e 64 anos possuem um dispositivo móvel com Android ou iOS, tendo assim um crescimento anual discreto no número de usuários devido a esta saturação. Nos Estados Unidos o número de pessoas entre 15 e 64 anos que possuem este tipo de dispositivo totalizam 310 milhões de pessoas, 78\% da população esta faixa etária \cite{flurry2012}.

Em uma pesquisa semelhante feita por acadêmicos da Universidade de Minho (localizada em Braga, Portugal) durante o ano de 2008, constatou-se que entre as 1225 pessoas de diferentes centros de ensino superior de Portugal que participaram da pesquisa, apenas 1\% das pessoas pesquisadas não possuem dispositivos móveis (\emph{Personal Digital Assitant} (PDA), smartphones). Levando-se em consideração o crescimento das plataformas móveis nos anos posteriores a este período, com o lançamento da primeira geração do iPhone no ano de 2008 em Portugal, e posteriormente a popularização da plataforma Android nos anos subsequentes, mostra que o uso de Celulares, smartphones, PDA's (e atualmente tablets) é popular nos meios acadêmicos deste país \cite{UsoTecnologiasMoveisEstudantesPortugueses}. Devido a Portugal ser um país desenvolvido e o Brasil ser um país em desenvolvimento, esta pesquisa pode não refletir a quantidade real da utilização nas instituições de ensino superior no Brasil ou da Unochapecó, mas demonstra que a utilização de dispositivos móveis em meios acadêmicos é comum, ficando até mesmo acima da média nacional. 

Por meio de uma pesquisa semelhante a realizada em Portugal, sendo esta aplicada na Unochapecó durante o desenvolvimento deste trabalho, será possível definir o percentual de usuários dentre os pesquisados que utilizam dispositivos móveis para acesso ao sistema acadêmico, e também definir quais as funcionalidades do Sistema Acadêmico Minha Uno são mais importantes para os usuários de cada perfil, podendo assim servir como base para o desenvolvimento da aplicação, atendendo diretamente as necessidades dos usuários do mesmo.

Tendo como ponto de partida a escolha das funcionalidades do sistema, é necessário definir as ferramentas utilizadas na implementação do aplicativo. Para isto, deve-se avaliar as três categorias de ferramentas de desenvolvimento para mobilidade multiplataforma existentes (Compilação Cruzada, Máquina Virtual e Webkit) para definir qual melhor atente das necessidades no desenvolvimento da aplicação. Cada classe de ferramentas possui um propósito, possuindo pontos positivos e negativos a serem levados em consideração durante o desenvolvimento da aplicação\cite{CrossPlatformMobileDevelopment2011}.

Para efetuar-se a integração entre aplicativos para dispositivos móveis e páginas web, são disponibilizados web services, que utilizando de protocolos como o \emph{Simple Object Access Protocol} (SOAP), \emph{Web Services Description Language} (WSDL), \emph{Universal Description Discovery and Integration} (UDDI) (ou outras tecnologias mais recentes como o \emph{JavaScript Object Notation} (JSON)), disponibilizam informações via Linguagem Extensível de Marcação (XML) para outras aplicações \cite{WebApplicationArchitecture}.  Tem-se como exemplo disto o Facebook, que disponibiliza uma \emph{Application programming interface} (API) de desenvolvimento, que extrai informações dos webservices, e também permite inserção de novas informações utilizando-se destas mesmas ferramentas \cite{Facebook}.



