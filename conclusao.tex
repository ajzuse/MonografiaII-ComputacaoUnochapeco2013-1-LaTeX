\chapter{Conclusão}

Este trabalho mostrou que é de interesse da comunidade acadêmica da instituição um aplicativo para dispositivos móveis, e que o desenvolvimento do mesmo também é possível. Este trabalho também mostra que mesmo quando não se possuindo acesso direto as bases de dados computacionais de uma instituição, a partir de visões ou conexões a tabelas, é possível obter-se as informações utilizando os meios onde as informações convencionalmente são disponibilizados, como neste caso, uma página da internet. Mesmo não sendo a melhor prática, com esforço e aplicação de conceitos é possível a obtenção dos dados desejados.

\section{Trabalhos Futuros}

Devido ao grau de complexidade do projeto não foi possível implementar todos os módulos do sistema acadêmico Minha Uno. Desta forma seria interessante a continuidade do projeto, implementando os módulos faltantes do perfil de graduação, e os outros perfis faltantes.

Para seguir as tendências de design atuais, seria muito interessante a adoção do design "flat" na aplicação, implementando desta forma uma interface gráfica mais atrativa aos usuários.

Como este trabalho depende do layout do arquivo HTML do sistema acadêmico atual, caso o sistema acadêmico sofra alterações de layout é necessária manutenção no servidor, adequando a extração dos dados ao novo layout do arquivo HTML.

Com o objetivo de a aplicação tornar-se opensource e poder atingir outras universidades, seria interessante padronizar a integração da aplicação e melhorar a sua abstração, possibilitando assim as universidades a partir do projeto base, sem muitas alterações no mesmo, poderem utilizar a aplicação para benefício dos seus acadêmicos.

Implementar notificações automáticas na aplicação, para avisar os usuários quando ocorre alguma alteração nos dados do mesmo, como o recebimento de um novo material ou uma nova nota cadastrada no sistema acadêmico.

Como dados são transmitidos entre o servidor e a aplicação, é necessário implementar medidas de segurança, para que os dados não possam ser interceptados no momento do sincronismo.