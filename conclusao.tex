\chapter{Conclusão}

Este trabalho mostrou que é de interesse da comunidade acadêmica da instituição um aplicativo para dispositivos móveis, e que o desenvolvimento do mesmo também é possível. Este trabalho também mostra que mesmo não possuindo acesso direto as bases de dados computacionais de uma instituição, a partir de visões ou conexões a tabelas, é possível obter-se as informações do sistema acadêmico utilizando os meios onde as informações convencionalmente são disponibilizadas, neste caso, uma página da internet. Mesmo não sendo a melhor prática, com esforço e aplicação de conceitos é possível a obtenção dos dados desejados.

Apesar da falta de apoio da diretoria de T.I. da instituição para fornecimento das informações e infra-estrutura, com a utilização de métodos de extração e o esforço para entender o funcionamento da página web do sistema acadêmico foi possível extrair as informações utilizando algumas vulnerabilidades do sistema acadêmico atual para isso. Sem estas vulnerabilidades exploradas (como por exemplo o sistema de login utilizado) o trabalho tornaria-se mais difícil, e quem sabe não fosse possível obter as informações do sistema acadêmico Minha Uno.

Além disso, apesar das diversas tentativas de envio do questionário a toda a instituição, para assim obter-se dados conclusivos sobre todos os módulos do sistema acadêmico, apenas os estudantes de graduação da Área de Ciências Exatas e Ambientais receberam o questionário, desta forma não sendo possível atingir toda a instituição, principalmente o corpo de docentes, que posteriormente ao questionário ficaram sabendo da existência do mesmo.

Apesar das dificuldades encontradas, foi possível efetuar o questionário com os acadêmicos e funcionários da Unochapecó, o qual trouxe respostas importantes para o desenvolvimento do trabalho, como as plataformas móveis utilizadas pelos acadêmicos e as formas de conexão a internet que os mesmos utilizam, além de quais as principais opções do sistema acadêmico foram utilizadas. Por meio deste questionário foi cumprido o primeiro objetivo proposto, o qual tratava do levantamento de informações do perfil acadêmicos por meio de um questionário e também foi cumprido o quarto objetivo que trata das plataformas móveis mais utilizadas pelos acadêmicos.

Após o levantamento dos requisitos, foi então definida a ferramenta de desenvolvimento utilizada por meio das plataformas apontadas pelos usuários do sistema acadêmico no questionário, desta forma se optando pelo Appcelerator Titanium como ferramenta de desenvolvimento. Desta forma o segundo objetivo proposto foi cumprido, escolhendo-se uma tecnologia de desenvolvimento que permite a implementação das opções mais relevantes do sistema acadêmico e atende as duas plataformas mais utilizadas pelos usuários do sistema Minha Uno.

Por meio da utilização de uma base SQLite na aplicação móvel, foi possível persistir os dados dos acadêmicos no dispositivos, podendo então o acadêmico consultar estes dados salvos anteriormente sem necessitar de uma conexão ativa a internet. Desta forma o quinto objetivo apresentado foi cumprido, disponibilizando as informações do acadêmico sem a necessidade de uma conexão com a internet.

Desta forma conclui-se que todos os objetivos propostos foram alcançados, mostrando que é possível desenvolver o aplicativo móvel para o sistema acadêmico Minha Uno, e que as mesmas informações podem ser utilizadas também para outros aplicativos, por meio da utilização dos servidor desenvolvido.

\section{Trabalhos Futuros}

Devido ao grau de complexidade do projeto não foi possível implementar todos os módulos do sistema acadêmico Minha Uno. Desta forma seria interessante a continuidade do projeto, implementando os módulos faltantes do perfil de graduação, e os outros perfis faltantes.

Para seguir as tendências de design atuais, seria muito interessante a adoção do design flat na aplicação, implementando desta forma uma interface gráfica mais atrativa aos usuários.

Como este trabalho depende do layout do arquivo HTML do sistema acadêmico atual, caso o sistema acadêmico sofra alterações de layout é necessária manutenção no servidor, adequando a extração dos dados ao novo layout do arquivo HTML.

Com o objetivo de a aplicação tornar-se opensource e poder atingir outras universidades, seria interessante padronizar a integração da aplicação e melhorar a sua abstração, possibilitando assim as universidades a partir do projeto base, sem muitas alterações no mesmo, poderem utilizar a aplicação para benefício dos seus acadêmicos.

Implementar notificações automáticas na aplicação, para avisar os usuários quando ocorre alguma alteração nos dados do mesmo, como o recebimento de um novo material ou uma nova nota cadastrada no sistema acadêmico.

Como dados são transmitidos entre o servidor e a aplicação, é necessário implementar medidas de segurança, para que os dados não possam ser interceptados no momento do sincronismo.