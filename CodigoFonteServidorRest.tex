\chapter{Icebreak REST Server}
\lstset{
    language=java,
    basicstyle=\scriptsize,
    upquote=true,
    aboveskip={1.5\baselineskip},
    columns=fullflexible,
    showstringspaces=false,
    extendedchars=true,
    breaklines=true,
    showtabs=false,
    showspaces=false,
    showstringspaces=false,
    frame=single,
    literate={á}{{\'a}}1 {ã}{{\~a}}1 {é}{{\'e}}1 {ç}{{\c{c}}}1 {Ç}{{\c{C}}}1 {Ã}{{\~A}}1 {Í}{{\'I}}1 {Ó}{{\'O}}1 {â}{{\^a}}1 {í}{{\'i}}1
}

Neste capítulo é apresentada a implementação do Icebreak REST Server. A implementação apresentada é similar a original, porém houve a necessidade de ser removido o tamanho do arquivo retornado no cabeçalho do HTML, o que pode ser visto na função \emph{sendResponse}. Foi destacada nesta função a linha comentada.

\begin{lstlisting}
/*                                                                                          */
/*    Copyright [2010] [System & Method A/S]                                                */
/*                                                                                          */
/*    Licensed under the Apache License, Version 2.0 (the "License");                       */
/*    you may not use this file except in compliance with the License.                      */
/*    You may obtain a copy of the License at                                               */
/*                                                                                          */
/*        http://www.apache.org/licenses/LICENSE-2.0                                        */
/*                                                                                          */
/*    Unless required by applicable law or agreed to in writing, software                   */
/*    distributed under the License is distributed on an "AS IS" BASIS,                     */
/*    WITHOUT WARRANTIES OR CONDITIONS OF ANY KIND, either express or implied.              */
/*    See the License for the specific language governing permissions and                   */
/*    limitations under the License.                                                        */
/*                                                                                          */
/*    Design - Niels Liisberg                                                               */
/*                                                                                          */

package edu.unochapeco.saa;

import java.io.*;
import java.net.*;
import java.util.*;
import java.text.*;
import java.io.FileInputStream;
import java.io.IOException;
import java.util.Properties;
import java.net.URL.*;
import java.net.URLDecoder;

/**
 * Super tiny HTTP serverside protocol for monolitic RESTservice applications
 * Simply drop the IceBreakRestServer jar file in your project ( classpath) and you are golden.
 *
 * A simple server looks like this:
 *
 *
 * <pre>
 * {@code
 * // Drop this jar-file into you project
 * import IceBreakRestServer.*;
 * import java.io.IOException;
 * public class Simple {
 *
 *   public static void main(String[] args) {
 *
 *     // Declare the IceBreak HTTP REST server class
 *     IceBreakRestServer rest;
 *
 *
 *     try {
 *
 *       // Instantiate it once
 *       rest  = new IceBreakRestServer();
 *
 *       while (true) {
 *
 *         // Now wait for any HTTP request
 *         // the "config.properties" file contains the port we are listening on
 *         rest.getHttpRequest();
 *
 *         // If we reach this point, we have received a request
 *         // now we can pull out the parameters from the query-string
 *         // if not found we return the default "N/A"
 *         String name = rest.getQuery("name", "N/A");
 *
 *         // we can now produce the response back to the client.
 *         // That might be XML, HTML, JSON or just plain text like here:
 *         rest.write("Hello world - the 'name' parameter is: " + name );
 *       }
 *     } catch (IOException ex) {
 *       System.out.println(ex.getMessage());
 *     }
 *   }
 * }
 * }
 * </pre>
 */


public class RestServer {

  private ServerSocket providerSocket = null;
  private Socket connection = null;
  private PrintWriter pw;
  private String ContentType;
  private String Status;
  private StringBuilder resp  = new StringBuilder(2048);
  private Boolean doFlush = false;
  private int Port ;
  private int Queue ;
  private InputStream in;

  /** This is the complete querysting including the resource. Just as you write it in your browser - you have to URL decode it or rather use getQuery to get paramter */
  public  String request;
  /** This is the contents sent by a POST  */
  public  String payload;
  /** This is the request type GET, POST, HEAD - your application have to responde coretly to this ( ore simply ignore it */
  public  String method ;
  /** This is the complete querysting after the resource as you write it in your browser - you have to URL decode it or rather use getQuery to get paramter */
  public  String queryStr;
  /** This is the name of the resource to run or get i.e. http://x/myApp.aspx/p1=abc it will return /myApp.aspx  */
  public  String resource;
  /** This is the version of the HTTP protocol requested   */
  public  String httpVer;
  /** Set this to true to get some system.out.print */
  public  Boolean debug = false; 

  /** This is the HTTP headers in the request. Use normal "Map" methods  */
  public  Map<String, String> header = new HashMap<String, String>();
  /** This is the HTTP quesystring parameters as map. Use normal "Map" methods or getQuery() method  */
  public  Map<String, String> parms  = new HashMap<String, String>();

  private void loadProps ( ) {
    Properties prop = new Properties();

    try {
      //load a properties file
      prop.load(new FileInputStream("config.properties"));
      Port  = Integer.parseInt(prop.getProperty("restserver.port","65000"));
      Queue = Integer.parseInt(prop.getProperty("restserver.queuesize", "10"));
    } catch (IOException ex) {
      // ex.printStackTrace();
    }
  }
  /**
	 * Contructor, returns an instance of the rest server
	 */
  public RestServer() {
    loadProps ();
  }

  /**
	 * Set the contents type of the HTTP contens. It has to conform
   * the mime type. By default it has the value of "text/plain; charset=utf-8"
	 * @param contents type string to set
	 */
  public void setContentType(String s) {
    ContentType = s;
  }
  /**
	 * Set the status of HTTP contens.
   * @see <a href="http://www.w3.org/Protocols/rfc2616/rfc2616-sec10.html">HTTP status codes</a>
	 * @param status string . by default the is "200 OK"
	 */
  public void setStatus(String s) {
    Status = s;
  }

  /**
	 * Set the TCP/IP port that you server is listening on. This is by default port 65000 and you can
   * set this value in the config.prperties file. Or you can set it programatically here but before issuing
   * a "getRequest()".
   * @param port TCP/IP port to listen on
	 */
  public void setPort(int port) {
    Port = port;
  }

  /**
	 * Set the TCP/IP queue depth for your HTTP server . This is by default port 10 and you can
   * set this value in the config.prperties file. Or you can set it programatically here but before issuing
   * a "getRequest()".
   * @param port TCP/IP port to listen on
	 */
  public  void setQueue(int queue) {
     Queue = queue;
  }

  /**
	 * Returns the parameter from the querystring with the name of "key". if the querystring
   * parameter was not fount it will return the default paramter
   * Note: key is case sensitive!!
   * @param Key - to return value for in the querystring
   * @param Default - when key is not found this wil be the default value
   * @return value of the querystring parameter
	 */
  public  String getQuery(String Key , String Default) {
     String temp = parms.get(Key);
     if (temp == null) return Default;
     return temp;
  }

  /**
	 * Returns the parameter from the querystring with the name of "key". if the querystring
   * parameter was not fount it will <code>null</code>
   * Note: key is case sensitive!!
   * @param Key - to return value for in the querystring
   * @return value of the querystring parameter
	 */
  public  String getQuery(String Key) {
     return parms.get(Key);
  }

  /**
	 * Just return a simple string with current timestamp in hh:mm:ss format
   * @return current time in hh:mm:ss format
	 */
  public String now () {
    String s;
    Format formatter;
    Date date = new Date();
    formatter = new SimpleDateFormat("hh:mm:ss");
    s = formatter.format(date);
    return s;
  }

  private static Map<String, String> getQueryMap(String query) {
     String[] params = query.split("&");  
     Map<String, String> map = new HashMap<String, String>();  
     for (String param : params)  {  
       int p = param.indexOf('=');
       if (p >= 0) {
         String name = param.substring( 0, p);
         String value = param.substring( p+1);
         String s = URLDecoder.decode(value);
         map.put(name, s);
       }
     }  
     return map;  
 }

 // This handles both windows <CR><LF> and mac/aix/linux <CR>
 // and returns both end of header and end of line sequence
 private int isEol(byte [] buf , int i) {
   if (buf[i] == 0x0d &&buf[i+1] == 0x0a) {
     if (buf[i+2] == 0x0d &&buf[i+3] == 0x0a) {
       return -4; // End Of header
     }
     return 2;
   }
   if (buf[i] == 0x0d ) {
     if (buf[i+1] == 0x0d ) {
       return -2; // End Of header
     }
     return 1;
   }
   if (buf[i] == 0x0a ) {
     if (buf[i+1] == 0x0a ) {
       return -2; // End Of header
     }
     return 1;
   }
   return 0;
}


 private void unpackRequest() throws IOException {

    byte buf [] = new byte[32768];
    in = connection.getInputStream();
    int read = in.read(buf);
    int len =0, pos =0, eol=0;
    header.clear();
    parms.clear();
    request = payload = method = queryStr = httpVer = resource = null;
    for (int i = 0; i < read && eol >= 0; i++) {
       eol  = isEol(buf , i);
       if (eol > 0) {
         // First line is the request. Now parse that partial
         if (request == null) {
            request =  new String(buf, pos  , len);
            String [] temp = request.split(" ");
            method = temp[0];
            queryStr = temp[1];
            httpVer = temp[2];
            int p = queryStr.indexOf('?');
            if (p>=0) {
              resource = queryStr.substring( 0, p);
              parms = getQueryMap(queryStr.substring( p+1));
            } else {
              resource = queryStr;
            }
         // Following lines are the header - put them into a map
         } else {
            String param  =  new String(buf, pos  , len);
            int p = param.indexOf(':');
            String name = param.substring( 0, p);
            String value = param.substring( p+1);
            header.put(name, value.trim());
         }
         len = 0;
         pos = i + eol;
         i+=eol-1;
       } else if (eol < 0) {
         pos = i + (-eol);
         payload = new String(buf, pos , read - pos);
       } else {
         len ++;
       }
    }

    // this is only for debugging
    if (debug) {
      System.out.println("resource: " + request);
      System.out.println("method: " + method);
      System.out.println("resource: " + resource);
      System.out.println("queryStr: " + queryStr);
      System.out.println("httpVer: " + httpVer);
      System.out.println("header  : " + header   );
      System.out.println("parms : " + parms  );
    }

  }

  private void sendResponse () {   
    
    pw.print("HTTP/1.1 " + Status + "\r\n" +
             "Connection: Keep-Alive\r\n" +
             "Accept: multipart/form-data\r\n"+
             "Accept-Encoding: multipart/form-data\r\n" +
             "Server: IceBreak Java Services\r\n" +
             "cache-control: no-store\r\n" +
             // LINHA COMENTADA POIS OCASIONAVA UM PROBLEMA DE FALTA DE 
             // DADOS NO JSON DE RETORNO
             //"Content-Length: " + Integer.toString(resp.length()) + "\r\n" +
             "Content-Type: " + ContentType + "\r\n" +
             "\r\n" + resp.toString());
    pw.flush();
  }

  /**
	 * This waits for the next HTTP request from the client
   *
	 */
  public void getHttpRequest () throws IOException {
    if (providerSocket == null) {
      providerSocket = new ServerSocket(Port , Queue);
    }
    if (doFlush) flush();

    connection = providerSocket.accept();
    pw = new PrintWriter(connection.getOutputStream());
    resp.setLength(0);
    unpackRequest();
    ContentType = "text/plain ;charset=utf-8";
    Status = "200 OK";
    doFlush = true;
  }

  /**
	 * write a string back a tring to the client. The complete result will be
   * send back to the client when you issue a "flush" or do the next "getHttpRequest()"
   * @param String - to send back to the client
	 */
  public void write(String s) {
    resp.append(s);
  }

  /**
	 * send back the complete response to the client. Now we are ready to wait for the next request by issue a "getHttpRequest()"
	 */
  public void flush() throws IOException {
    sendResponse ();
    connection.close();
    doFlush = false;
  }
}
\end{lstlisting}
